% ------------------------------------------------------------------------
% AMS-LaTeX Paper
% ------------------------------------------------------------------------
% Submitted:      Trans.Amer.Math.Soc. in February 1995
% Final Version:  July 1995
% Accepted:       June 1995
% ------------------------------------------------------------------------
% This is my doc
%%%%%%%%%%%%%%%%%%%%%%%%%%%%%%%%%%%%%%%%%%%%%%%%%%%%%%%%%%%%%%%%%%%%%%%%%%

\documentclass[10pt]{article}
\usepackage[centertags]{amsmath}
\usepackage{amsfonts}
\usepackage{amssymb}
\usepackage{amsthm}
\usepackage{newlfont}
\usepackage[T1]{fontenc}
\usepackage[ansinew]{inputenc}
\usepackage{graphicx}
\usepackage{color}
\usepackage[colorlinks]{hyperref}

% Over-full v-boxes on even pages are due to the \v{c} in author's name
\vfuzz2pt % Don't report over-full v-boxes if over-edge is small

% THEOREM Environments ---------------------------------------------------
 \newtheorem{thm}{Theorem}[subsection]
 \newtheorem{cor}[thm]{Corollario}
 \newtheorem{lem}[thm]{Lemma}
 \newtheorem{prop}[thm]{Proposizione}
 \theoremstyle{definition}
 \newtheorem{defn}[thm]{Definizione}
 \theoremstyle{remark}
 \newtheorem{rem}[thm]{Oss.}
 \numberwithin{equation}{subsection}
% MATH -------------------------------------------------------------------
 \DeclareMathOperator{\RE}{Re}
 \DeclareMathOperator{\IM}{Im}
 \DeclareMathOperator{\ess}{ess}
 \newcommand{\eps}{\varepsilon}
 \newcommand{\To}{\longrightarrow}
 \newcommand{\s}{\mathcal{S}}
 \newcommand{\A}{\mathcal{A}}
 \newcommand{\E}{\mathcal{E}}
 \newcommand{\h}{\mathcal{H}}
 \newcommand{\M}{\mathcal{M}}
 \newcommand{\W}{\mathcal{W}}
 \newcommand{\X}{\mathcal{X}}
 \newcommand{\Y}{\mathcal{Y}}
 \newcommand{\Z}{\mathcal{Z}}
 \newcommand{\C}{\mathcal{C}}
 \newcommand{\BOP}{\mathbf{B}}
 \newcommand{\BH}{\mathbf{B}(\mathcal{H})}
 \newcommand{\KH}{\mathcal{K}(\mathcal{H})}
 \newcommand{\Real}{\mathbb{R}}
 \newcommand{\Complex}{\mathbb{C}}
 \newcommand{\Field}{\mathbb{F}}
 \newcommand{\RPlus}{\Real^{+}}
 \newcommand{\Polar}{\mathcal{P}_{\s}}
 \newcommand{\Poly}{\mathcal{P}(E)}
 \newcommand{\EssD}{\mathcal{D}}
 \newcommand{\Lom}{\mathcal{L}}
 \newcommand{\States}{\mathcal{T}}
 \newcommand{\abs}[1]{\left\vert#1\right\vert}
 \newcommand{\set}[1]{\left\{#1\right\}}
 \newcommand{\seq}[1]{\left<#1\right>}
 \newcommand{\norm}[1]{\left\Vert#1\right\Vert}
 \newcommand{\essnorm}[1]{\norm{#1}_{\ess}}

%%% ----------------------------------------------------------------------
\begin{document}

%\subjclass{Primary 47A15; Secondary 46A32, 47D20}
%\keywords{...}
%\dedicatory{}
%\commby{Daniel J. Rudolph}
\title{Simple model of atomic decay}
\author{Gabriele Filosofi}
\date{May 2006}

%%% ----------------------------------------------------------------------
\maketitle

%%% ----------------------------------------------------------------------
\begin{abstract}
Firstly a simplified model of atomic decay is introduced. Then we
make some investigation about the free and forced response of a
related differential system.
\end{abstract}

%%% ----------------------------------------------------------------------
\section{Introduction}
\smallskip

%Consideriamo il seguente modello semplificato di "decadimento
%atomico". Il modello prevede la catena di $N=3$ elementi $X_k$
%($k=1\dots N$), dove $X_1$ � l'elemento capostipite della catena
%(quello con massimo numero di massa) e $X_i$ decade in $X_{i+1}$
%con costante di tempo $\tau_i=\frac{1}{r_i}>0$. $x_i$ sia la
%quantit� di atomi del tipo $X_i$. Nell'approssimazione di modello
%continuo $x_i$ � una funzione continua del tempo. Inizialmente
%ipotizziamo una quantit� $x_0>0$ dell'elemento $X_0$, mentre
%$x_2(0)=x_3(0)=0$. Il sistema risolvente � formato da 3 equazioni
%differenziali ordinarie del primo ordine lineari omogenee

Let's consider a simplified first order linear model for atomic
decay. It is composed of $N$ different kind of atomic elements,
$X_i$ (always $i=1\dots N$), where $X_1$ is the top element of the
chain (that with the greatest mass number) and $X_i$ decays to
$X_{i+1}$ with time constant $\tau_i=\frac{1}{r_i}>0$. Let $x_i(t)$
be the number of type $X_i$ atoms at a given time. In the continous
approximation of the model $x_i$ is a continous function of time.


\section{Three atoms model}
\smallskip

Let's assume $N=3$ and an initial number $x_i^0>0$ of type $X_i$
atoms. The solving system is made up by three first order
homogeneous ordinary differential equations

\bigskip
\[
\left\{\begin{array}{l}
\dot{x}_1=-r_1x_1\\
\dot{x}_2=r_1x_1-r_2x_2\\
\dot{x}_3=r_2x_2\\
\end{array}\right.
\]

or, with obvious notation,

\begin{equation}\label{e:eq00}
\dot{\vec{x}}(t)=A\vec{x}(t) \quad\quad t>0; \quad\quad \vec{x}(0)=\vec{x}_0 \\
\end{equation}

with

\[
A\equiv\left(\begin{array}{ccc}
-r_1 & 0 & 0 \\
r_1 & -r_2 & 0\\
0 & r_2 & 0 \end{array}\right) \quad\quad
\vec{x}_0\equiv\left(\begin{array}{c}
x_1^0\\x_2^0\\x_3^0\end{array}\right)
\]
\bigskip

The matrix $A$ is singular. Having all zero elements above the
principal diagonal its eigenvalues can be read immediately along the
diagonal, $\lambda_1=-r_1$, $\lambda_2=-r_2$ and $\lambda_3=0$. They
are all real and distinct numbers. The general solution is

\begin{equation}\label{e:eq01}
\vec{x}(t)=C_1e^{-r_1t}\vec{\eta}^{(1)}+C_2e^{-r_2t}\vec{\eta}^{(2)}+C_3\vec{\eta}^{(3)}\\
\end{equation}

We have to determine the eigenvectors $\vec{\eta}^{(i)}$ by solving
$(A-\lambda_iI)\vec{\eta}^{(i)}=0$, and find the constants $C_i$ by
imposing the initial conditions.
\goodbreak For eigenvalue $\lambda_1=-r_1$ we have

\bigskip
\[
\left(\begin{array}{ccc}
0 & 0 & 0 \\
r_1 & r_1-r_2 & 0\\
0 & r_2 & r_1 \end{array}\right) \left(\begin{array}{c}
\eta_1^{(1)}\\
\eta_2^{(1)}\\
\eta_3^{(1)}\end{array}\right) = \vec{0}\quad\quad\Rightarrow
\vec{\eta}^{(1)} = \left(\begin{array}{c}
\tfrac{r_2}{r_1}-1\\1\\-\tfrac{r_2}{r_1}\end{array}\right)
\]

\goodbreak For eigenvalue $\lambda_2=-r_2$

\bigskip
\[
\left(\begin{array}{ccc}
r_2-r_1 & 0 & 0 \\
r_1 & 0 & 0\\
0 & r_2 & r_2 \end{array}\right) \left(\begin{array}{c}
\eta_1^{(2)}\\
\eta_2^{(2)}\\
\eta_3^{(2)}\end{array}\right) = \vec{0}\quad\quad\Rightarrow
\vec{\eta}^{(2)} = \left(\begin{array}{c}
0\\-1\\1\end{array}\right)
\]

\goodbreak For eigenvalue $\lambda_3=0$

\bigskip
\[
\left(\begin{array}{ccc}
-r_1 & 0 & 0 \\
r_1 & -r_2 & 0\\
0 & r_2 & 0 \end{array}\right) \left(\begin{array}{c}
\eta_1^{(3)}\\
\eta_2^{(3)}\\
\eta_3^{(3)}\end{array}\right) = \vec{0}\quad\quad\Rightarrow
\vec{\eta}^{(3)} = \left(\begin{array}{c}
0\\0\\1\end{array}\right)
\]
\bigskip

From the initial conditions

\bigskip
\[
C_1\left(\begin{array}{c}
\tfrac{r_2}{r_1}-1\\1\\-\tfrac{r_2}{r_1}\end{array}\right)
 + C_2\left(\begin{array}{c}
0\\-1\\1\end{array}\right)
 + C_3\left(\begin{array}{c}
0\\0\\1\end{array}\right) = \left( \begin{array}{c}
x_1^0\\x_2^0\\x_3^0\end{array}\right)
\]
\bigskip

we find $C_1=\frac{x_1^0r_1}{r_2-r_1}$,
$C_2=\frac{x_1^0r_1}{r_2-r_1}-x_2^0$, $C_3=x_1^0+x_2^0+x_3^0$.
\goodbreak Thus, the solution is

\bigskip
\[
\left\{\begin{array}{l}
x_1(t)=x_1^0e^{-r_1t}\\\\
x_2(t)=\frac{x_1^0r_1}{r_2-r_1}e^{-r_1t}+(x_2^0-\frac{x_1^0r_1}{r_2-r_1})e^{-r_2t}\\\\
x_3(t)=x_1^0+x_2^0+x_3^0-\frac{x_1^0r_1}{r_2-r_1}e^{-r_1t}-(x_2^0-\frac{x_1^0r_1}{r_2-r_1})e^{-r_2t}\\
\end{array}\right.
\]
\bigskip

The picture shows how all these quantities vary with time, having
$\vec{x}_0=(x_0,0,0); \tau_1=300; \tau_2=1200$.

\begin{figure}[h]\label{fig:fig1}
 \begin{center}
   \includegraphics[width=3in]{fig1}\\
 \end{center}
 \caption{Simplified model of atomic decay - System component's time evolution ($\tau_1=300; \tau_2=1200$).}
\end{figure}

\newpage
\section{Digression about closed chains - homogeneous}
\smallskip

In the atomic decay model discussed above, the mass flows from
$x_1$ through $x_N$ and stay there indefinitely. Now, with a
little modification we make the chain cyclic

\bigskip
\[
\left\{\begin{array}{l}
\dot{x}_1=r_3x_3-r_1x_1\\
\dot{x}_2=r_1x_1-r_2x_2\\
\dot{x}_3=r_2x_2-r_3x_3\\
\end{array}\right.
\]

i.e. "mass" can go back to $x_1$ trough the rate $r_3$. In matrix
notation,

\begin{equation}\label{e:eq02}
\dot{\vec{x}}(t)=A\vec{x}(t) \quad\quad t>0; \quad\quad \vec{x}(0)=\vec{x}_0 \\
\end{equation}

\[
A\equiv\left(\begin{array}{ccc}
-r_1 & 0 & r_3 \\
r_1 & -r_2 & 0\\
0 & r_2 & -r_3 \end{array}\right) \quad\quad
\vec{x}_0\equiv\left(\begin{array}{c}
x_1^0\\x_2^0\\x_3^0\end{array}\right) \quad\quad
\vec{r}\equiv\left(\begin{array}{c}
r_1\\r_2\\r_3\end{array}\right)
\]
\bigskip

It is an isolated system, no "mass" sources or sinks being involved
(i.e. the system is homogeneous).
\goodbreak A graphical representation of the system is sketched below.
Each arrow is directed with the "mass" flow and it is labeled with
the transfer rate from the originating variable to the next one.

\begin{figure}[h]\label{fig:fig7}
 \begin{center}
   \includegraphics[width=1.5in]{fig7}\\
 \end{center}
 \caption{Isolated three-variables unidirectional closed chain.}
\end{figure}

This model has no obvious physical significance, indeed we
investigate such a system just for the pleasure to do it.
\goodbreak Note that

\[
\dot{x}_1+\dot{x}_2+\dot{x}_3=0\\
\]

This holds because the system is an isolated system, and this
implies that the scalar quantity

\[
\E\equiv x_1+x_2+x_3=x_1^0+x_2^0+x_3^0=const.\\
\]

it is a time invariant. Without any forcing terms into the equations
we're sure the system will stay bounded all the time. i.e it is
asintotically stable for $t\rightarrow +\infty$.

Also, we assume $r_i>0$. The matrix $A$ is nonsingular. To find
the eigenvalues, let's solve the characteristic equation

\[
|A-\lambda I|=\left|\begin{array}{ccc}
-r_1-\lambda & 0 & r_3 \\
r_1 & -r_2-\lambda & 0\\
0 & r_2 & -r_3-\lambda \end{array}\right| =
-(\lambda+r_1)(\lambda+r_2)(\lambda+r_3)+r_1r_2r_3=0
\]

This gives

\begin{equation}\label{e:eq10}
\lambda\cdot[\lambda^2+\lambda(r_1+r_2+r_3)+r_1r_2+r_2r_3+r_3r_1]=0
\end{equation}

An eigenvalue is $\lambda_1=0$, while $\lambda_2$ and $\lambda_3$
are the roots of the quadratic factor

\[
\lambda_{2,3}=\frac{-S\pm\sqrt{r_1^2+r_2^2+r_3^2-2P}}{2}\\
\]

with $S\equiv r_1+r_2+r_3$ and $P\equiv r_1r_2+r_2r_3+r_3r_1$. What
about the sign of the quantity under square root? From the Schwartz
inequality we know that

\[
r_1^2+r_2^2+r_3^2-P>0\\
\]

Nevertheless $\vec{r}=(r_1,r_2,r_3)$ can be chosen such that

\[
r_1^2+r_2^2+r_3^2-2P\leq0\\
\]

so any cases could take place. Just let's mention the following
cases. When $\vec{r}=(\rho,\rho,\rho)$ the term under the square
root is negative and $\lambda_{2,3}=\frac{-3\pm j\sqrt{3}}{2}\rho$.
\goodbreak If two of the rates are equal to $\rho$ and the third
rate is equal to $4\rho$, then $\lambda_2=\lambda_3=-3\rho$.
\goodbreak Finally, if one of the rates is greater than twice the
sum of the others, $\lambda_{2,3}$ are real and distinct - the
last statement coming from the inequalities

\[
r_1^2+r_2^2+r_3^2\geq 2r_1r_2+r_3^2 > 2P\quad\Leftrightarrow\quad r_3^2 > 2P-2r_1r_2=2r_2r_3+2r_3r_1=2r_3(r_1+r_2)\\
\]

Note that the assumption $r_i>0$ is very important for the time
evolution of the system. In fact it implies that, whether
eigenvalues  $\lambda_{2,3}$ will be real or complex, their real
part will have to be negative, so that the system not only is
stable, but takes a constant value in the long term
$t\rightarrow+\infty$, i.e.

\begin{equation}\label{e:eq03}
\lim_{t\rightarrow+\infty}\vec{x}(t)=\vec{x}_\infty=const.
\end{equation}

Now, in order to move toward an explicit solution we have to
determine the eigenvectors by solving
$(A-\lambda_iI)\vec{\eta}^{(i)}=0$.

\subsection{Eigenvalue $\lambda_1=0$}
\smallskip

For the zero eigenvalue we have

\bigskip
\[
\left(\begin{array}{ccc}
-r_1 & 0 & r_3 \\
r_1 & -r_2 & 0\\
0 & r_2 & -r_3 \end{array}\right) \left(\begin{array}{c}
\eta_1^{(1)}\\
\eta_2^{(1)}\\
\eta_3^{(1)}\end{array}\right) = \vec{0}\quad\quad\Rightarrow
\vec{\eta}^{(1)} = \left(\begin{array}{c}
\frac{r_3}{r_1}\\\frac{r_3}{r_2}\\1\end{array}\right)
\]
\bigskip

This eigenvector plays an important role because it is associated
with the steady state value of the system trajectory,
$\vec{x}_\infty$, as we'll see in the next paragraph.

\subsection{$\lambda_{2,3}$ real and distinct eigenvalues}
\smallskip

\goodbreak For real and distinct eigenvalues $\lambda_{2,3}$ the general solution is

\begin{equation}\label{e:eq04}
\vec{x}(t)=C_1\vec{\eta}^{(1)}+C_2e^{\lambda_2t}\vec{\eta}^{(2)}+C_3e^{\lambda_3t}\vec{\eta}^{(3)}\\
\end{equation}

We have

\bigskip
\[
\left(\begin{array}{ccc}
-r_1-\lambda_{2,3} & 0 & r_3 \\
r_1 & -r_2-\lambda_{2,3} & 0\\
0 & r_2 & -r_3-\lambda_{2,3} \end{array}\right)
\left(\begin{array}{c}
\eta_1^{(2,3)}\\
\eta_2^{(2,3)}\\
\eta_3^{(2,3)}\end{array}\right) = \vec{0}\quad\quad\Rightarrow
\vec{\eta}^{(2,3)} = \left(\begin{array}{c}
r_3\\\frac{r_1r_3}{r_2+\lambda_{2,3}}\\r_1+\lambda_{2,3}\end{array}\right)
\]
\bigskip

To make things easier, let's consider the special case
$\vec{r}=(\rho,\tfrac{1}{6}\rho,\tfrac{1}{3}\rho)$,
$\vec{x}_0=(x_0,0,0)$, so that $\lambda_2=-\frac{2}{3}\rho$ and
$\lambda_3=-\frac{5}{6}\rho$. Using this assumptions the eigenvector
associated with the zero eigenvalue becomes

\[
\vec{\eta}^{(1)} = \left(\begin{array}{c}
1\\6\\3\end{array}\right)
\]

\goodbreak For eigenvalue $\lambda_2=-\frac{2}{3}\rho$ we have

\bigskip
\[
\vec{\eta}^{(2)} = \left(\begin{array}{c}
1\\-2\\1\end{array}\right)
\]
\bigskip

\goodbreak For eigenvalue $\lambda_3=-\frac{5}{6}\rho$

\bigskip
\[
\vec{\eta}^{(3)} = \left(\begin{array}{c}
2\\-3\\1\end{array}\right)
\]
\bigskip

Constants $C_i$ have to be determined by imposing the initial
conditions

\bigskip
\[
C_1\left(\begin{array}{c} 1\\6\\3\end{array}\right)
 + C_2\left(\begin{array}{c}
1\\-2\\1\end{array}\right) + C_3\left(\begin{array}{c}
2\\-3\\1\end{array}\right) = \left(
\begin{array}{c} x_1^0\\0\\0\end{array}\right)
\]
\bigskip

We get $C_1=\frac{1}{10}x_1^0$, $C_2=-\frac{3}{2}x_1^0$,
$C_3=\frac{6}{5}x_1^0$. The solution is

\bigskip
\[
\left\{\begin{array}{l}
x_1(t)=\frac{x_1^0}{10}(1-15e^{-\frac{2\rho t}{3}}+24e^{-\frac{5\rho t}{6}})\\\\
x_2(t)=\frac{x_1^0}{5}(3+15e^{-\frac{2\rho t}{3}}-18e^{-\frac{5\rho t}{6}})\\\\
x_3(t)=\frac{x_1^0}{10}(3-15e^{-\frac{2\rho t}{3}}+12e^{-\frac{5\rho t}{6}})\\\\
\end{array}\right.
\]

The picture below shows how all the quantities vary in time
($\rho=1$).

\begin{figure}[h]\label{fig:fig3}
 \begin{center}
   \includegraphics[width=3in]{fig3}\\
 \end{center}
 \caption{Time course for the three system components with real and distinct eigenvalues ($\tau_1=\tfrac{1}{3}; \tau_2=2; \tau_3=1$).}
\end{figure}

For generic initial conditions we would have had
$C_1=\tfrac{1}{10}(x_1^0+x_2^0+x_3^0)$,
$C_2=\tfrac{1}{4}(-3x_1^0-x_2^0+3x_3^0)$,
$C_3=\tfrac{1}{5}(6x_1^0+x_2^0-4x_3^0)$. It's worth to note that the
coefficient related with the zero eigenvalue is proportional to
$\E$. The same held in the atomic decay model discussed above. In
our example the steady state constant is

\begin{equation}
\vec{x}_\infty=C_1\vec{\eta}^{(1)}=\frac{\E}{10}\left(\begin{array}{c}
1\\
6\\
3\end{array}\right)
\end{equation}

%The next picture shows the long term values $x_i^\infty$ normalized
%to $\E$.
%\begin{figure}[h]\label{fig:fig6}
% \begin{center}
%   \includegraphics[width=3in]{fig6}\\
% \end{center}
% \caption{The long term "mass" content of the three variables.}
%\end{figure}


\bigskip
\newpage
\subsection{$\lambda_{2,3}$ complex eigenvalues}
\smallskip

\goodbreak For complex eigenvalue $\lambda_2=\alpha+j\beta$ we have

\bigskip
\[
\left(\begin{array}{ccc}
-r_1-\alpha-j\beta & 0 & r_3 \\
r_1 & -r_2-\alpha-j\beta & 0\\
0 & r_2 & -r_3-\alpha-j\beta \end{array}\right)
\left(\begin{array}{c}
\eta_{1R}+j\eta_{1I}\\
\eta_{2R}+j\eta_{2I}\\
\eta_{3R}+j\eta_{3I}\end{array}\right)
\]
\bigskip

This is equivalent to solve the real system

\bigskip
\[
\left(\begin{array}{cccccc}
-\alpha_1 & \beta & 0 & 0 & r_3 & 0\\
\beta & \alpha_1 & 0 & 0 & 0 & -r_3\\
r_1 & 0 & -\alpha_2 & \beta & 0 & 0\\
0 & -r_1 & \beta & \alpha_2 & 0 & 0\\
0 & 0 & r_2 & 0 & -\alpha_3 & \beta\\
0 & 0 & 0 & -r_2 & \beta & \alpha_3\\
\end{array}\right)
\left(\begin{array}{c}
\eta_{1R}\\
\eta_{1I}\\
\eta_{2R}\\
\eta_{2I}\\
\eta_{3R}\\
\eta_{3I}\end{array}\right)=\vec{0}
\]
\bigskip

where $\alpha_i\equiv r_i+\alpha$. \goodbreak But let's consider the
special case $\vec{r}=(\rho,\rho,\rho)$, so that
$\lambda=\frac{-3\pm j\sqrt{3}}{2}\rho$.
\goodbreak With this assumption the eigenvector coupled with the zero eigenvalue
becomes

\[
\vec{\eta}^{(1)} = \left(\begin{array}{c}
1\\1\\1\end{array}\right)
\]

\goodbreak For eigenvalue $\lambda_2=\frac{-3+j\sqrt{3}}{2}\rho$

\[
\left(\begin{array}{ccc}
\tfrac{1-j\sqrt{3}}{2}\rho & 0 & \rho \\
\rho & \tfrac{1-j\sqrt{3}}{2}\rho & 0\\
0 & \rho & \tfrac{1-j\sqrt{3}}{2}\rho
\end{array}\right) \left(\begin{array}{c}
\eta_{1R}+j\eta_{1I}\\
\eta_{2R}+j\eta_{2I}\\
\eta_{3R}+j\eta_{3I}\end{array}\right) =
\vec{0}\quad\quad\Rightarrow \vec{\eta}^{(2)} =
\left(\begin{array}{c}
1\\-\tfrac{1+j\sqrt{3}}{2}\\-\tfrac{1-j\sqrt{3}}{2}\end{array}\right)
\]

We know that $\lambda_3=\lambda_2^*$ and
$\vec{\eta}^{(3)}=(\vec{\eta}^{(2)})^*$.
From the initial conditions

\bigskip
\[
C_1\left(\begin{array}{c} 1\\1\\1\end{array}\right)
 + C_2\left(\begin{array}{c}
1\\-\tfrac{1+j\sqrt{3}}{2}\\-\tfrac{1-j\sqrt{3}}{2}\end{array}\right)
+ C_3\left(\begin{array}{c}
1\\-\tfrac{1-j\sqrt{3}}{2}\\-\tfrac{1+j\sqrt{3}}{2}\end{array}\right)
= \left( \begin{array}{c} x_1^0\\0\\0\end{array}\right)
\]
\bigskip

and we obtain $C_1=C_2=C_3=\frac{x_1^0}{3}$. The solution is

\bigskip
\[
\vec{x}(t)= \tfrac{x_1^0}{3}\left(\begin{array}{c}
1\\1\\1\end{array}\right)
 + \tfrac{x_1^0}{3}\left(\begin{array}{c}
1\\-\tfrac{1+j\sqrt{3}}{2}\\-\tfrac{1-j\sqrt{3}}{2}\end{array}\right)e^{\frac{-3+j\sqrt{3}}{2}\rho
t}
 + \tfrac{x_1^0}{3}\left(\begin{array}{c}
1\\-\tfrac{1-j\sqrt{3}}{2}\\-\tfrac{1+j\sqrt{3}}{2}\end{array}\right)e^{\frac{-3-j\sqrt{3}}{2}\rho
t}
\]
\bigskip

or

\bigskip
\[
\left\{\begin{array}{l}
x_1(t)=\frac{x_1^0}{3}(1+2e^{-\frac{3\rho t}{2}}\cos\frac{\sqrt{3}\rho t}{2})\\\\
x_2(t)=\frac{x_1^0}{3}(1+\sqrt{3}e^{-\frac{3\rho t}{2}}\sin\frac{\sqrt{3}\rho t}{2}-e^{-\frac{3\rho t}{2}}\cos\frac{\sqrt{3}\rho t}{2})\\\\
x_3(t)=\frac{x_1^0}{3}(1-\sqrt{3}e^{-\frac{3\rho t}{2}}\sin\frac{\sqrt{3}\rho t}{2}-e^{-\frac{3\rho t}{2}}\cos\frac{\sqrt{3}\rho t}{2})\\
\end{array}\right.
\]
\bigskip

The picture below shows how all the quantities vary in time, with
$\tau_1=\tau_2=\tau_3=1$.

\begin{figure}[h]\label{fig:fig4}
 \begin{center}
   \includegraphics[width=3in]{fig4}\\
 \end{center}
 \caption{Time course for the three system components with two complex eigenvalues ($\tau_1=\tau_2=\tau_3=1$).}
\end{figure}

\newpage
\subsection{$\lambda_{2,3}$ real and equal eigenvalues}
\smallskip

Let's consider the special case $\vec{r}=(\rho,\rho,4\rho)$. In this
case $\lambda_2=\lambda_3=-3\rho\equiv\lambda$. Now the eigenvector
associated with the zero eigenvalue is

\[
\vec{\eta}^{(1)} = \left(\begin{array}{c}
4\\4\\1\end{array}\right)
\]

From linear algebra we know that the double eigenvalue $\lambda$
can have one or two independent eigenvectors. In this case it has
only one

\bigskip
\[
\left(\begin{array}{ccc}
2\rho & 0 & 4\rho\\
\rho & 2\rho & 0\\
0 & \rho & -\rho \end{array}\right) \left(\begin{array}{c}
\eta_1\\
\eta_2\\
\eta_3\end{array}\right) = \vec{0}\quad\quad\Rightarrow \vec{\eta}
= \left(\begin{array}{c} -2\\1\\1\end{array}\right)
\]
\bigskip

With these two only independent vectors we cannot fit the solution
with the initial conditions. In fact, we need to introduce an
auxiliary constant vector $\vec{\theta}$ into the the general
solution

\begin{equation}\label{e:eq05}
\vec{x}(t)=C_1\vec{\eta}^{(1)}+C_2e^{\lambda t}\vec{\eta}+C_3e^{\lambda t}(t\vec{\eta}+\vec{\theta})\\
\end{equation}

Replacing the above expression in the equation
$\dot{\vec{x}}=A\vec{x}$, with simple matrix algebra we find that
$\vec{\theta}$ must satisfy $(A-I\lambda)\vec{\theta}=\vec{\eta}$.
After some calculation we arrive at

\[
\vec{\theta} = \left(\begin{array}{c} -1\\1\\0\end{array}\right)
\]

Using the initial conditions, $C_1=\frac{x_1^0}{9}$,
$C_2=-\frac{x_1^0}{9}$, $C_3=-\frac{x_1^0}{3}$. The final solution
is

\bigskip
\[
\left\{\begin{array}{l}
x_1(t)=\frac{x_1^0}{9}[4+e^{-3\rho t}(6t+5)]\\\\
x_2(t)=\frac{x_1^0}{9}[4-e^{-3\rho t}(3t+4)]\\\\
x_3(t)=\frac{x_1^0}{9}(1-4e^{-3\rho t})\\\\
\end{array}\right.
\]
\bigskip

The picture below shows how all the quantities vary in time, with
$\tau_1=\tau_2=1$, $\tau_3=\frac{1}{4}$.

\begin{figure}[h]\label{fig:fig5}
 \begin{center}
   \includegraphics[width=3in]{fig5}\\
 \end{center}
 \caption{Time course for the three system components with two complex eigenvalues ($\tau_1=\tau_2=\tau_3=1$).}
\end{figure}



\newpage
\section{Nonhomogeneous systems}
\smallskip

Now we may ask what would happen if forcing functions were
introduced into the system. In order to simplify calculations we
consider a two variables (i.e. a second order) system. After a quick
derivation of the homogeneous solution we will find the Green
function kernel, with which the reader is assumed to be familiar,
and get the particular solution for a couple of examples.

\bigskip
\subsection{Two variables - homogeneous}
\smallskip

\bigskip
\[
\left\{\begin{array}{l}
\dot{x}_1=r_2x_2-r_1x_1\\
\dot{x}_2=r_1x_1-r_2x_2\\
\end{array}\right.
\]

or, in matrix notation,

\begin{equation}\label{e:eq06}
\dot{\vec{x}}(t)=A\vec{x}(t) \quad\quad t>0; \quad\quad \vec{x}(0)=\vec{x}_0 \\
\end{equation}

\[
A\equiv\left(\begin{array}{cc}
-r_1 & r_2 \\
r_1 & -r_2\end{array}\right) \quad\quad
\vec{x}_0\equiv\left(\begin{array}{c} x_1^0\\x_2^0\end{array}\right)
\]
\bigskip

To find the eigenvalues, let's solve the characteristic equation

\[
|A-\lambda I|= (\lambda+r_1)(\lambda+r_2)-r_1r_2=0
\]

Eigenvalues are $\lambda_1=0$ and
$\lambda_2=-(r_1+r_2)\equiv\lambda$. The general solution is

\begin{equation}\label{e:eq07}
\vec{x}(t)=C_1\vec{\eta}^{(1)}+C_2e^{\lambda t}\vec{\eta}^{(2)}\\
\end{equation}

Now, in order to move toward an explicit solution we have to
determine the eigenvectors by solving
$(A-\lambda_iI)\vec{\eta}^{(i)}=0$.

\bigskip
\[
\left(\begin{array}{cc}
-r_1 & r_2\\
r_1 & -r_2\end{array}\right) \left(\begin{array}{c}
\eta_1^{(1)}\\
\eta_2^{(1)}\end{array}\right) = \vec{0}\quad\quad\Rightarrow
\vec{\eta}^{(1)} = \left(\begin{array}{c}
\frac{r_2}{r_1}\\1\end{array}\right)
\]
\bigskip

\bigskip
\[
\left(\begin{array}{cc}
r_2 & r_2\\
r_1 & r_1\end{array}\right) \left(\begin{array}{c}
\eta_1^{(2)}\\
\eta_2^{(2)}\end{array}\right) = \vec{0}\quad\quad\Rightarrow
\vec{\eta}^{(2)} = \left(\begin{array}{c} 1\\-1\end{array}\right)
\]
\bigskip

Constants $C_i$ have to be determined by imposing the initial
conditions

\bigskip
\[
C_1\left(\begin{array}{c} \tfrac{r_2}{r_1}\\1\end{array}\right)
 + C_2\left(\begin{array}{c}
1\\-1\end{array}\right) = \left(
\begin{array}{c} x_1^0\\x_2^0\end{array}\right)
\]
\bigskip

We get $C_1=\frac{r_1(x_1^0+x_2^0)}{r_1+r_2}$,
$C_2=\frac{r_1x_1^0-r_2x_2^0}{r_1+r_2}$. The solution is

\bigskip
\[
\left\{\begin{array}{l}
x_1(t)=\frac{r_2(x_1^0+x_2^0)}{r_1+r_2}+\frac{r_1x_1^0-r_2x_2^0}{r_1+r_2}e^{\lambda t}\\\\
x_2(t)=\frac{r_1(x_1^0+x_2^0)}{r_1+r_2}-\frac{r_1x_1^0-r_2x_2^0}{r_1+r_2}e^{\lambda t}\\\\
\end{array}\right.
\]

\begin{figure}[h]\label{fig:fig10}
 \begin{center}
   \includegraphics[width=4in]{fig10}\\
 \end{center}
 \caption{Time course for the two system's components.}
\end{figure}


\bigskip
\subsection{Two variables - nonhomogeneous}
\smallskip
Now the system looks like

\[
\left\{\begin{array}{l}
\dot{x}_1=r_2x_2-r_1x_1+f_1\\
\dot{x}_2=r_1x_1-r_2x_2+f_2\\
\end{array}\right.
\]

where $f_i(t)\in C^1[0,+\infty)$ are the forcing functions. Taking
the derivative of the first equation and using
$\dot{x}_1+\dot{x}_2=f_1+f_2$ we obtain the equivalent second order
nonhomogeneous ODE with nonhomogeneous (i.e. non zero) BCs

\[
\left\{\begin{array}{l}
\ddot{x}_1-\lambda\dot{x}_1=g(t)\quad\quad 0\leq t<+\infty\\
x_1(0)=x_1^0\\
\dot{x}_1(0)=r_2x_2^0-r_1x_1^0\equiv K
\end{array}\right.
\]

where $\lambda\equiv -r_1-r_2$ and $g\equiv
r_2(f_1+f_2)+\dot{f}_1$. This is called a Cauchy problem (or
Initial Value Problem), the solution to which is assumed to have
the form

\begin{equation}\label{e:eq001}
x_1(t)=x_1^{(o)}+x_1^{(p)}\\
\end{equation}

where $x_1^{(o)}$ is the solution to the homogeneous problem
($g(t)\equiv 0$) with nonhomogeneous BCs, and $x_1^{(p)}$ is a
particular solution to the nonhomogeneous problem with homogeneous
(i.e. zero) BCs. From the previous paragraph we know that

\begin{equation}\label{e:eq002}
x_1^{(o)}=x_1^0+\frac{K}{\lambda}(e^{\lambda t}-1)\\
\end{equation}

\bigskip
\subsubsection{Green Function method}
\smallskip

The Green function method serves to find $x_1^{(p)}$. The Green
function $G(t,\xi)$ is an integral kernel and must satisfy the
homogeneous equation for $t\neq\xi$, thus
\[
G(t,\xi)=\left\{\begin{array}{l}
A_1(\xi)e^{\lambda t}+B_1(\xi) \ \ \ per \ t<\xi\\
A_2(\xi)e^{\lambda t}+B_2(\xi) \ \ \ per \ t>\xi
\end{array}\right.
\]

Following the standard procedure, we apply homogeneous BCs as well
as continuity and jump conditions to $G(t,\xi)$ in order to get the
unknown functions $A_1$, $B_1$, $A_2$ and $B_2$

\begin{equation}\label{e:eq10}
G(0,\xi)=0\quad\Rightarrow A_1(\xi)+B_1(\xi)=0\\
\end{equation}
\begin{equation}\label{e:eq11}
\frac{\partial G}{\partial t}|_{t=0}=0\quad\Rightarrow A_1(\xi)=0\quad\Rightarrow B_1(\xi)=0\\
\end{equation}
\begin{equation}\label{e:eq12}
G(\xi^+,\xi)-G(\xi^-,\xi)=0\quad\Rightarrow A_2(\xi)e^{\lambda\xi}+B_2(\xi)=0\\
\end{equation}
\begin{equation}\label{e:eq13}
\frac{\partial G}{\partial t}|_{t=\xi^-}^{t=\xi^+}=1 \quad\Rightarrow A_2(\xi)=\frac{1}{\lambda}e^{-\lambda\xi}\quad\Rightarrow B_2(\xi)=-\frac{1}{\lambda}\\
\end{equation}

So

\[
G(t,\xi)=\left\{\begin{array}{l}
0 \ \ \ \ \ \ \ \ \ \ \ \ \ \ \ \ \ \ \ for \ t<\xi\\
\frac{1}{\lambda}[e^{\lambda(t-\xi)}-1] \ \ \ for \ t\geq\xi
\end{array}\right.
\]

Then, the particular solution for the nonhomogeneous system is

\begin{equation}\label{e:eq08}
x_1^{(p)}(t)=\int_{0}^{+\infty}G(t,\xi)g(\xi)d\xi= \tfrac{1}{\lambda}\int_{0}^{t}[e^{\lambda(t-\xi)}-1]g(\xi)d\xi\\
\end{equation}

The solution to the original Cauchy problem is

\begin{equation}\label{e:eq08c}
x_1(t)=x_1^0+\frac{K}{\lambda}(e^{\lambda t}-1)+\tfrac{1}{\lambda}\int_{0}^{t}[e^{\lambda(t-\xi)}-1]g(\xi)d\xi\\
\end{equation}

One can repeat exactly the same for the component $x_2(t)$ (the only
change affects the integration constants in the homogeneous solution
and the forcing function in the particular solution). Putting all
together the final result is

\bigskip
\[
\left\{\begin{array}{l}
x_1(t)=\frac{r_2(x_1^0+x_2^0)}{r_1+r_2}+\frac{r_1x_1^0-r_2x_2^0}{r_1+r_2}e^{\lambda t}+\tfrac{1}{\lambda}\int_{0}^{t}[e^{\lambda(t-\xi)}-1][r_2(f_1(\xi)+f_2(\xi))+\dot{f}_1(\xi)]d\xi\\\\
x_2(t)=\frac{r_1(x_1^0+x_2^0)}{r_1+r_2}-\frac{r_1x_1^0-r_2x_2^0}{r_1+r_2}e^{\lambda t}+\tfrac{1}{\lambda}\int_{0}^{t}[e^{\lambda(t-\xi)}-1][r_1(f_1(\xi)+f_2(\xi))+\dot{f}_2(\xi)]d\xi\\
\end{array}\right.
\]


\bigskip
\subsubsection{Variation of Parameters method}
\smallskip

An alternative method to find a particular solution to the second
order nonhomogeneous equations is that of variation of parameters.
Basically it consists in applying the formula

\begin{equation}\label{e:eq08e}
y^{(p)}(t)=-y_1(t)\int\frac{y_2(\xi)g(\xi)}{W(y_1,y_2)}d\xi+y_2(t)\int\frac{y_1(\xi)g(\xi)}{W(y_1,y_2)}d\xi\\
\end{equation}

where $y_1$ and $y_2$ represents a fundamental set of solutions for
the homogeneous problem, and $W$ their Wronskian. For our problem we
can put $y_1=e^{\lambda t}$ and $y_2=1$ (Do not worry about which of
our two solutions is $y_1$ and which one is $y_2$. It doesn�t
matter).

\bigskip
\[
W(y_1,y_2)\equiv\left|\begin{array}{cc}
y_1 & y_2\\
\dot{y}_1 & \dot{y}_2\end{array}\right| =\left|\begin{array}{cc}
e^{\lambda t} & 1\\
\lambda e^{\lambda t} & 0\end{array}\right| =-\lambda e^{\lambda t}
\]
\bigskip
Substitution in \ref{e:eq08e} and integration gives ($y=x_1$)

\begin{equation}\label{e:eq08f}
x_1^{(p)}(t)=\tfrac{1}{\lambda}\int_{0}^{t}[e^{\lambda(t-\xi)}-1]g(\xi)d\xi\\
\end{equation}

This coincides with the previously stated result. We now look at a
series of examples.

\newpage
\subsection{Example I}
\smallskip

Let's take the example

\[
\left\{\begin{array}{l}
f_1(t)=Du(t-t_0)\\
f_2(t)=-Du(t-t_0)
\end{array}\right.
\]

with $t_0>0$ and $t\geq0$. Here $u(t)$ denotes the unit (or
Heaviside) step function. We have

\[
g(t)\equiv r_2(f_1+f_2)+\dot{f}_1=D\delta(t-t_0)\\
\]

\begin{align}
x_1^{(p)}(t)=\tfrac{1}{\lambda}\int_{0}^{t}[e^{\lambda(t-\xi)}-1]D\delta(\xi-t_0)d\xi\\
=\tfrac{D}{\lambda}[e^{\lambda(t-t_0)}-1]u(t-t_0)\\
=-\tfrac{D}{r_1+r_2}[e^{-(r_1+r_2)(t-t_0)}-1]u(t-t_0)
\end{align}

\begin{figure}[h]\label{fig:fig8}
 \begin{center}
   \includegraphics[width=2.5in]{fig8}\\
 \end{center}
 \caption{Example I. The particular solution ($\lambda=-1,-2,-3; D=3$)}
\end{figure}



\newpage
\subsection{Example II}
\smallskip

%Consider another example
%\[
%\left\{\begin{array}{l}
%f_1(t)=D\sin(\omega t)\\
%f_2(t)=0
%\end{array}\right.
%\]
%We have
%\[
%g(t)=r_2D\sin(\omega t)+D\omega\cos(\omega t)\\
%\]
%\begin{align}
%x_1(t)=\tfrac{D}{\lambda}\int_{0}^{t}[e^{\lambda(t-\xi)}-1][r_2\sin(\omega\xi)+\omega\cos(\omega\xi)]d\xi\\
%=\tfrac{D}{\lambda}e^{\lambda t}[r_2I_1+\omega
%I_2]_0^t-\tfrac{D}{\lambda}\int_{0}^{t}[r_2\sin(\omega\xi)+\omega\cos(\omega\xi)]d\xi\\
%\end{align}
%\[
%I_1=\int e^{-\lambda\xi}\sin(\omega\xi)d\xi = -\frac{\lambda\sin(\omega\xi)+\omega\cos(\omega\xi)}{\lambda^2+\omega^2}e^{-\lambda\xi}\\
%\]
%\[
%I_2=\int e^{-\lambda\xi}\cos(\omega\xi)d\xi = \frac{\omega\sin(\omega\xi)-\lambda\cos(\omega\xi)}{\lambda^2+\omega^2}e^{-\lambda\xi}\\
%\]
%\begin{equation}
%x_1(t)=\tfrac{D}{\lambda(\lambda^2+\omega^2)}[(\omega^2-r_2\lambda)\sin(\omega
%t)+\omega(\lambda+r_2)(e^{\lambda t}-1)\cos(\omega t)] - \tfrac{D}{\lambda\omega}[\sin(\omega t)+r_2(1-\cos(\omega t))]\\
%\end{equation}

Consider another example

\[
\left\{\begin{array}{l}
f_1(t)=0\\
f_2(t)=D\sin(\omega t)
\end{array}\right.
\]

We have

\[
g(t)=r_2D\sin(\omega t)\\
\]

\begin{align}
x_1(t)=\tfrac{D}{\lambda}\int_{0}^{t}[e^{\lambda(t-\xi)}-1]r_2\sin(\omega\xi)d\xi\\
=\tfrac{r_2D}{\lambda}e^{\lambda
t}[I_1]_0^t-\tfrac{r_2D}{\lambda}\int_{0}^{t}\sin(\omega\xi)d\xi
\end{align}

where

\[
I_1=\int e^{-\lambda\xi}\sin(\omega\xi)d\xi = -\frac{\lambda\sin(\omega\xi)+\omega\cos(\omega\xi)}{\lambda^2+\omega^2}e^{-\lambda\xi}\\
\]

By substitution we find

\begin{equation}
x_1^{(p)}(t)=\tfrac{r_2D}{\lambda(\lambda^2+\omega^2)}[\lambda\sin(\omega
t)+\omega\cos(\omega t)-\omega e^{\lambda t}] - \tfrac{r_2D}{\lambda\omega}[1-\cos(\omega t)]\\
\end{equation}

\begin{figure}[h]\label{fig:fig9}
 \begin{center}
   \includegraphics[width=3in]{fig9}\\
 \end{center}
 \caption{Example II. The particular solution ($\lambda=-1,-2,-3;D=3$)}
\end{figure}

\newpage
\section{Table of atomic decay chains}
\smallskip

\begin{figure}[h]\label{fig:fig11}
 \begin{center}
   \includegraphics[width=3in]{fig11}\\
 \end{center}
 \caption{Table of atomic decay chains.}
\end{figure}

% ------------------------------------------------------------------------
%\subsection*{Ringraziamenti}
%I miei ringraziamenti a ...
% ------------------------------------------------------------------------
% -------------------------------------------------------------------
\end{document}
% ------------------------------------------------------------------------
